% Standardinkluderingsfil
\input{standard}

\ifpdf
  \DeclareGraphicsExtensions{.pdf, .jpg, .tif, .png}
  \pdfinfo{            
    /Title  (System Wide Requirements)
    /Author (PUM-grupp 1)
  }
\else
  \DeclareGraphicsExtensions{.eps, .jpg}
\fi

\title{Distribuerad wiki \\ System-WideRequirements \\ Version 1.1}
\author{PUM-grupp 1}
\date{\today}

\begin{document}

\maketitle\thispagestyle{empty}

\newpage

{\centering \Large{Dokumenthistorik\\}}

\vspace{10pt}
\begin{tabularx}{\textwidth}{ |l|l|X|l|l| }
  \hline
    \textbf{version} & \textbf{datum} & \textbf{utförda ändringar} & \textbf{utförda av} & \textbf{granskad} \\
	\hline 
  0.1 & 2009-02-12 &  Dokumentet är påbörjat, all engelsk text kommer från OpenUP & Erik & Alla  \\
  \hline
  1.0 & 2009-03-03 & Dokumentet är klart & Erik & Alla \\
  \hline
  1.1 & 2009-03-09 & Uppdaterad efter opponering & Erik & Linus \\
  \hline 
\end{tabularx}

\newpage

\setcounter{tocdepth}{3}
\tableofcontents
\newpage

\section{Introduktion}
Dokumentet ska innehålla krav på applikationen som är svåra att fånga upp i användarfallen. Dokumentet är uppdelat enligt olika aspekter på applikationen. 

%\section{Systemomfattande funktionskrav}
%Här redogörs funktionskrav som är systemomfattande. Krav som inte fångas upp i användarfallen. Det kan röra protokoll och arkitektur 

\section{Systemegenskaper}
Krav på applikationen som ställs utifrån den egenskap som vi önskar på systemet. 

\subsection{Användbarhet}
Målet är att användaren ska kunna starta en ny wiki, ansluta till en wiki och redigera olika wiki sidor utan tekiskt stöd. Det ska mätas och utvärderas med en användargrupp där minst 70 \% ska klara uppgifterna. 

\subsection{Tillförlitlighet} 
Applikationens tillförlitlighet beror främst på hur data hanteras. Användaren ska kunna lita på att den data som användaren lägger till kommer vara oförändrat när det läses vid ett senare tillfälle. Det kan mätas med tester på skillnaden mellan in och utdata, det får inte skillja något mellan in och utdata. En stor del ansvaret kommer ligga på den versionshanterare som används så därför bör även den testas och utvärderas. 

\subsection{Prestanda}
Systemet är inte prestandakrävande, bandbreddskrävande och kräver inte snabba svarstider. Programmet är gjort så att data sparas på den lokala datorns hårddisk där access- och skrivtid är snabb (beror på datorns och hårddiskens snabbhet). Synkroniseringen med andra användare sker under ytan utan att användaren påverkas. Fel eller konflikter rörande synkroniseringen med andra användare ska inte behöva åtgärdas omedelbart och inte heller störa användaren i pågående arbete. Uppstart av applikationen ska max ta 3 sekunder.

\subsection{Support och påbyggnadsmöjligheter} 
Applikationen ska byggas modulbaserat. Det är viktigt att olika delar kan uppdateras för sig just för att vår applikation bygger på flera olika moduler. Systeminstallationen ska gå via en pakethanterar i linux.

%This section indicates any requirements that will enhance the supportability or maintainability of the system being built, including adaptability and upgrading, compatibility, configurability, scalability and requirements regarding system installation, level of support and maintenance.

\section{Systemgränssnitt}
Krav på gränssnitt mellan hårdvara, mjukvara och mot användaren. 

\subsection{Användargränsnitt}
I detta avsnitt behandlas krav rörande användargränssnittet. 

\subsubsection{Utseende och känsla}
Applikationen ska i redigeringsläget kännas som en ordbehandlare, formatering av text ska visas direkt i redigeringsfönstret till skillnad från traditionell wiki där taggar används. Gränssnittet ska upplevas som effektivt och enkelt, knappar till vanliga funktioner och snabbkommandon ska finnas. Synkronisering mot andra användare ska ske i bakgrunden utan att påverka användarens arbete. 

\subsubsection{Layout och navigering}
Användaren ska kunna nå stora delar av systemet med endast ett fåtal navigeringar. Max tre navigeringshandlingar för att utföra enklare funktionern till exempel editera eller lägg till ny artikel. Mer avancerade inställningar som att starta en ny wiki kan vara en 10 navigeringshandlingar. Navigeringen ska vara intuitiv och efterlikna operativsystemet som programmet körs på. Beteendemönster ska kännas igen från andra textredigeringsprogram och Webbläsare. Applikationen har främst två lägen, ett för visning och ett för redigering och från varje visningssida som man har rätt att redigera ska man enkelt kunna byta till redigeringsläge. När man redigerat klart och godkänner ändringarna så återvänder man till visningssläget.

%\subsubsection{Följdriktighet} %Consistency
%Consistency in the user interface enables users to predict what will happen. This section states requirements on the use of mechanisms to be employed in the user interface. This applies both within the system and with other systems and can be applied at different levels: navigation controls, screen areas sizes and shapes, placements for entering / presenting data, terminology.

\subsubsection{Användaranpassning och kundanpassning}
Det är ett krav att vi ska släppa applikationen som open source. Open source är en otroligt effektivt metod för kundanpassning. Koden är öppen så att det är fritt för kunden att göra förändringar och förbättringar. 

\subsection{Gränssnitt mot externa system}
Inga externa system tas hänsyn till som skulle kräva eget gränssnitt. 

\subsubsection{Mjukvarugränssnitt}
Applikationen kommer i huvudsak kodas i python. För extern kommunikation används distributionssystemet Bazaar. Bazaar tillhandahåller en komplett uppsättning funktioner för versionhantering. Bazaar är skrivet i Python och är därför enkelt att integrera.

För användargränssnitt kommer wxPython att användas. wxPython är ett bibliotek för att tillhandahålla grafiska användargränssnitt till Pythonprogram och kan anpassa sig till flera operativsystem.

\subsubsection{Hårdvarugränsnitt}
Python tillhandahåller abstraktioner för den grundläggande operativsystemsfunktionaliteten.

\subsubsection{Kommunikationsgräsnitt}
%Här kan jag tänka mig en mer avancerad förklaring hur vi har tänkt oss p2p, krypetering osv
Kommunikation mellan klienter sker med hjälp av TLS över TCP/IP. För detta krävs tillgång till ett nätverksgrässnitt på det lokala systemet samt en Python-specifik kommunikationsmodul som tillhandahåller TLS över TCP/IP. All form av binär data kan skickas som en ström över TLS.

För att underlätta kommunikationen mellan två klienters distributionssystem kommer vi tunnla datatrafiken dem emellan genom vår TLS-länk. Detta är särskilt lämpligt om en användare ligger bakom en brandvägg, där distributionssystemet i normala fall endast skulle kunna verka enkelriktat.

Klienter kommunicerar över nätverket med ett gränssnitt som möjliggör automatisk och komplett propagering av ändringar, administration av medlemsskap i nätverket och autenticering av medlemmar. 

%Describe any communications interfaces to other systems or devices such as local area networks, remote serial devices, and so on.

\section{Systembegränsningar}
Applikationen kommer skrivas i python. Det finns flera tredjeparts delar bland annat Bazaar och wxPython. Applikationen kommer utvecklas för linux till att börja med men då pythonkod kan interpreteras av Windows och Mac OSX så kommer versioner eventuellt släppas till dessa system också i mån av tid. Slutprodukten är tänkt att kunna utnyttjas av en grupp på upp till 10 personer kring en gemensam distribuerad wiki.

\section{Övrig systeminformation} %Vet inte om det finns en bätte översättning

\subsection{Licenskrav}
Projektet ska släppas som open source. Vilka licenser vi väljer beror på om vi utnyttjar redan befintlig kod med restriktiva licenser.

\subsection{Rättsligt, upphovsrätt och andra meddelanden}
Det ligger som förslag att lägga programmet på en idell förening. Detta för att förenkla upphovsrättsliga frågor när programmet ska vidareutvecklas. Annars kommer det hamna på oss som utvecklare och vår kund. 

%\subsection{Applicable Standards}
%This section describes by reference any applicable standards and the specific sections of any such standards that apply to the system being described. For example, this could include legal, quality and regulatory standards, industry standards for usability, interoperability, internationalization, operating system compliance, and so forth.

\section{Systemdokumentation}
Systemdokumentationen kommer finnas på projektets hemsida i form av en wiki där användarna kan medverka. Dokumentering av kod och övrig dokumentation för utvecklarna bör följa med koden. Användarmanuler kommer följa med när man installerar programmet via en paktethanterare. % Vem är ansvarig för punken?

\end{document}
