% Standardinkluderingsfil
\input{standard}

\ifpdf
  \DeclareGraphicsExtensions{.pdf, .jpg, .tif, .png}
  \pdfinfo{            
    /Title  (Glossary)
    /Author (PUM-grupp 1)
  }
\else
  \DeclareGraphicsExtensions{.eps, .jpg}
\fi

\title{Distribuerad wiki \\ Glossary \\ Version 1.1 }

\author{PUM-grupp 1}
\date{\today}

\begin{document}

\maketitle\thispagestyle{empty}

\newpage

{\centering \Large{Dokumenthistorik\\}}

\vspace{10pt}
\begin{tabularx}{\textwidth}{ |l|l|X|l|l| }
  \hline
    \textbf{version} & \textbf{datum} & \textbf{utförda ändringar} & \textbf{utförda av} & \textbf{granskad} \\
	\hline 1.0 & 2009-02-12 &  Första versionen klar för inlämning  & Alla & Alla   \\
  \hline 1.1 & 2009-03-09 & Fler termer inlagda. & Martin & Erik\\
  \hline
\end{tabularx}

\newpage

\setcounter{tocdepth}{2}
\tableofcontents
\newpage


\section{Inledning}

Många av termerna inom detta område som kan vara svåra att förstå och behöver klargöras. Därför har denna ordlista producerats för att kunna få en gemensam bild av dessa termer. Ordlistan är sorterad i bokstavsordning.

\section{Ordlista}
\begin{itemize}
        \item \textbf{Bazaar}
        \\Bazaar är det distribueringssystem som används för att distribuera filer mellan olika klienter i nätverket.
	\item \textbf{Branch}
	\\I ett projekt vill man oftast att olika personer ska kunna arbeta på samma filer samtidigt. För att inte behöva vänta på att andra personer ska arbeta klart på filen kan man skapa en kopia av hela projektet så att modifikationer kan ske parallellt. Detta kallas för en branch.
	
	\item \textbf{Centraliserat versionshanteringssystem}
	\\Ett centraliserat versionshanteringssystem bygger på en central server där allt arbete finns lagrat. När en användare har hämtat ner filer från servern och ändrat på dem kan denne ladda upp filerna till servern igen (se Commit) och ändringarna sparas på servern så att andra användare sedan kan hämta ner dessa.
	
	\item \textbf{Commit}
	\\Varje gång en användare känner att han har gjort ändringar som är värda att spara kan används kommandot commit för att registrera ändringarna. Oftast inkluderas också ett litet meddelande om vad ändringarna gäller. Ändringarna skickas då till ett repository där motsvarande filer uppdateras.
	
	\item \textbf{Distribuerat versionshanteringssystem}
	\\Ett distribuerat versionshanteringssystem skiljer från det centraliserade i huvudsak på det sättet att det inte finns någon central server. Istället har varje användare en lokal kopia av projektet och kan sedan hämta nya versioner av filer från varandra.
	
	\item \textbf{Fork}
	\\Att skapa en branch från ett projekt kallas ofta för att forka.
	
	\item \textbf{Klient}
	\\En klient är en instans av programvaran som körs aktivt på någon
plattform.	
	
	\item \textbf{Konflikt}
	\\Om två eller fler användare ändrat i samma stycke i en fil och använt kommandot commit så uppstår en konflikt när någon användare sedan får hem två av dessa olika versioner av filen. Versionshanteringssystemet vet då inte vilken version den ska använda och signalerar att en konflikt har skett. Detta måste då lösas manuellt via granskning av konflikterande data.

	\item \textbf{Merge}
	\\Detta sker när en användare försöker hämta ner nya versioner av filer från andra användares repositories. Versionshanteringssystemet försöker då att sammanfoga ändringarna som skett i de olika filerna.

	\item \textbf{Nyckel}
	\\En datamängd som används för signering, verifiering och säker kommunikation.
			
	\item \textbf{Repository}
	\\En repository är ett ställe där filerna är sparade. I den centraliserade systemet är detta på servern och på det distrubuerade har varje användare en egen repository på den lokala datorn.

	\item \textbf{Resolve}
	\\Att lösa en konflikt kallas för resolve.
	
	\item \textbf{Revision}
	\\En revision eller version är en ändring av något slag i filerna tillhörande projektet. Varje commit skapar en ny revision.

        \item \textbf{TCP/IP}
        \\Två kommunikationsprotokoll som tillsammans utgör grunden för robust och tillförlitlig kommunikation mellan datorer. Den kommunicerade dataströmmen kommer garanterat fram oskadd och i rätt ordning.

        \item \textbf{TLS}
        \\Ett krypterat kommunikationsgränssnitt som används för säker kommunikation mellan datorer. Det använder TCP/IP som underliggande kommunikationsprotokoll.
	
	\item \textbf{Wiki} \\
	En applikation där artiklarna kan redigeras av användarna själva.
\end{itemize}
\end{document}
